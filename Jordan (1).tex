\documentclass[12pt, a4paper,oneside, UTF8]{ctexart}
\usepackage[dvipsnames]{xcolor}
\usepackage{amsmath}   % 数学公式
\usepackage{amsthm}    % 定理环境
\usepackage{amssymb}   % 更多公式符号
\usepackage{graphicx}  % 插图
\usepackage{mathrsfs}  % 数学字体
\usepackage{enumitem}  % 列表
\usepackage{geometry}  % 页面调整
\usepackage{physics}
\usepackage[colorlinks,linkcolor=black]{hyperref}
\usepackage[english]{babel}
\usepackage{lmodern}
\usepackage{braket}
\usepackage{quiver}
\usepackage{listings}
\graphicspath{ {flg/},{../flg/}, {config/}, {../config/} }  % 配置图形文件检索目录
\linespread{1.5} % 行高

% 页码设置
\geometry{top=25.4mm,bottom=25.4mm,left=20mm,right=20mm,headheight=2.17cm,headsep=4mm,footskip=12mm}

% 设置列表环境的上下间距
\setenumerate[1]{itemsep=5pt,partopsep=0pt,parsep=\parskip,topsep=5pt}
\setitemize[1]{itemsep=5pt,partopsep=0pt,parsep=\parskip,topsep=5pt}
\setdescription{itemsep=5pt,partopsep=0pt,parsep=\parskip,topsep=5pt}

% 定理环境
% ########## 定理环境 start ####################################
\theoremstyle{definition}
\newtheorem{defn}{\indent 定义}
\newtheorem{def}{\indent 定义}

\newtheorem{lemma}{\indent 引理}    % 引理 定理 推论 准则 共用一个编号计数
\newtheorem{lem}[lemma]{\indent 引理} % 也可以用这个命令
\newtheorem{thm}[lemma]{\indent 定理}
\newtheorem{corollary}[lemma]{\indent 推论}
\newtheorem{cor}[lemma]{\indent 推论} % 也可以用这个命令
\newtheorem{criterion}[lemma]{\indent 准则}

\newtheorem{proposition}[lemma]{\indent 命题}
\newtheorem{prop}[lemma]{\indent 命题} % 也可以用这个命令
\theoremstyle{plain}
\newtheorem{example}{\indent {例}} % 绿色文字的 例 ,不需要就去除\color{SeaGreen}{}
\newtheorem{eg}[example]{\indent {例}} % 也可以用这个命令
\newtheorem*{rmk}{\indent 注}
\newtheorem*{rem}[rmk]{\indent 注} % 也可以用这个命令
\newtheorem*{practice}{\indent 练习}
\newtheorem*{prac}[practice]{\indent 练习} % 也可以用这个命令

% 两种方式定义中文的 证明 和 解 的环境:
% 缺点:\qedhere 命令将会失效【技术有限,暂时无法解决】
\renewenvironment{proof}{\par\textbf{证明.}\;}{\qed\par} %该行的具体作用是什么
\newenvironment{solution}{\par{\textbf{解.}}\;}{\qed\par}

% 缺点:\bf 是过时命令,可以用 textb f等替代,但编译会有关于字体的警告,不过不影响使用【技术有限,暂时无法解决】
%\renewcommand{\proofname}{\indent\bf 证明}
%\newenvironment{solution}{\begin{proof}[\indent\bf 解]}{\end{proof}}
% ######### 定理环境 end  #####################################

% ↓↓↓↓↓↓↓↓↓↓↓↓↓↓↓↓↓ 以下是自定义的命令  ↓↓↓↓↓↓↓↓↓↓↓↓↓↓↓↓

% 用于调整表格的高度  使用 \hline\xrowht{25pt}
\newcommand{\xrowht}[2][0]{\addstackgap[.5\dimexpr#2\relax]{\vphantom{#1}}}

% 表格环境内长内容换行
\newcommand{\tabincell}[2]{\begin{tabular}{@{}#1@{}}#2\end{tabular}}

% 使用\linespread{1.5} 之后 cases 环境的行高也会改变,重新定义一个 ca 环境可以自动控制 cases 环境行高
\newenvironment{ca}[1][1]{\linespread{#1} \selectfont \begin{cases}}{\end{cases}}
% 和上面一样
\newenvironment{vx}[1][1]{\linespread{#1} \selectfont \begin{vmatrix}}{\end{vmatrix}}

\def\d{\textup{d}} % 直立体 d 用于微分符号 dx
\def\R{\mathbb{R}} % 实数域
\newcommand{\bs}[1]{\boldsymbol{#1}}    % 加粗,常用于向量
\newcommand{\ora}[1]{\overrightarrow{#1}} % 向量

% 数学 平行 符号
\newcommand{\pll}{\kern 0.56em/\kern -0.8em /\kern 0.56em}

% 用于空行\myspace{1} 表示空一行 填 2 表示空两行  
\newcommand{\myspace}[1]{\par\vspace{#1\baselineskip}}
\author{chengd23 }
\date{November 2023}
\usepackage{titlesec} 
\usepackage{graphicx}
\usepackage{float} %84-88有何具体用处,为什么在这个位置

\title{\textbf{Jordan标准型}}
\author{程笛}
\date{2024.7.8}
\begin{document}
\maketitle
回顾线性空间的定义:
线性空间 $V$是一个配有数乘运算的加法Abel群:
\[
\mathbb{K} \times V \to V, \ (k,v) \mapsto kv
\] 
并且数乘运算满足一些相容条件.我们推广线性空间的定义, 我们称一个类似上述的集合为模, 如果将要求的 $\mathbb{K}$换为环.(以下环都指含单位元的交换环)

\begin{defn}
    环 $R$上的集合 $M$被称为 $R$-模,如果满足
    \begin{enumerate}
        \item 存在加法运算,是关于加法运算的Abel群
        \item 数乘运算
        \[
        R \times M \to M, \ (r,m) \mapsto rm
        \]
    \begin{itemize}
        \item $(r+s)m = rm+sm, \forall r,s \in R, m \in M$
        \item $(rs)m = r(sm), \forall r,s \in R, m \in M$
        \item $r(m+n)= rm+rn,\forall r \in R, m,n \in M$
        \item $1m = m, \forall m \in M$
    \end{itemize}
    \end{enumerate}
\end{defn}

如果 $R$也是域, 那么 $R$-模就是 $R$线性空间.


模 $M$的子模 $N$指一个子集, 其满足对数乘封闭.即 $RN \subseteq N$.

\begin{example}
    环 $R$本身可以看为 $R$-模,此时其上的子模就是理想.
\end{example}
    \begin{example}
    $R$-模 $M$,$I$是 $R$的理想,则
    \[
    IM :=\{ x_1e_1+ \cdots +x_ne_{n}\, ;\, x_{i}\in I, e_{i}\in M ,n\in \mathbb{N}\}
    \]
    是子模.
\end{example}


类似线性空间, 我们也可以定义模同态和商模.


\begin{thm}
    $M,N$是 $R$-模,对于同态
    \[
    \varphi:M \to N
    \]
    则 $\operatorname{ker}\varphi$是 $M$的子模, $\operatorname{Im}\varphi$也是 $R$-模.
    有诱导同构
    \[
    M / \operatorname{ker}\varphi \cong \operatorname{Im}\varphi
    \]
\end{thm}
\begin{proof}
    $$m \in \operatorname{ker}\varphi \iff \varphi(m)= 0 \iff \varphi(rm) = r\varphi(m) = 0, \forall r \in R \iff \operatorname{ker}\varphi\text{是子模}$$
    $$n \in \operatorname{Im}\varphi \iff \exists m \in M, \varphi(m) = n $$从而 $rn = \varphi(rm)\in \operatorname{Im}\varphi$

    同构证明: 映射
    \[
    [m] = m+\operatorname{ker}\varphi \mapsto \varphi(m)
    \]显然是满射($m$原像的所有元素形成等价类 $[m]$),又 $\varphi(m) = 0 \iff m \in \operatorname{ker}\varphi$,因此 $[m] = [0]$,从而是单射, 从而是同构.
\end{proof}

类似线性空间, 在模上我们可以试着找"基", 方便起见, 此后我们只谈论环 $R$是主理想整环,且 $M$的基向量有限的情况.

\begin{defn}
     $R$-模 $M$是自由的, 如果存在一个有限子集 $\{ e_1, \ldots ,e_{m} \}$, $M$中的任何一个向量均可被该子集的元素唯一地线性表出.这一子集即为 $M$的一个基, 称 $M$的秩(rank)为 $m$.
\end{defn}

 \begin{example}
    有限维线性空间是自由的.
 \end{example}

 \begin{defn}
    $R$-模 $M$称为有限生成的, 如果存在有限子集 $W = \{ w_1, \ldots ,w_{m} \}$,它们的线性组合生成整个 $M$.
 \end{defn}
\begin{example}
    整数环 $\mathbb{Z}$视为 $\mathbb{Z}$-模是由 $1$有限生成的, $\mathbb{Z}/ \braket{n}$视为 $\mathbb{Z}$模也是有限生成的, 但不是自由的.
\end{example}
 

对于自由模 $M$, 上述$W$生成 $M$, 则有同态

    $$\pi _{w} : R^{m} \to M, \ \ (x_{1},x_{2},\dots,x_{m}) \mapsto x_{1}w_{1}+x_{2}w_{2}+\dots x_{m}w_{m}$$
    
    是满射.从而有 $R^{m}/ \operatorname{ker}\pi_{w} \cong M$.特别的,如果是单射,那么 $R^{m} \cong M$



\begin{defn}
    通过对有限个 $R$-模 $M_{i}$的直积定义运算
    \[
    (v_1, \ldots ,v_{m})+(w_1, \ldots ,w_{m}) =(v_1+w_1, \ldots , v_{m}+w_{m}), v_{i},w_i\in M_i
    \]
    由此定义了模直和,对应线性空间的外直和.

    对于子模的直和, 则与线性空间子空间直和一致, 即
    $$(v_1, \ldots ,v_{m}) \mapsto v_1+ \cdots +v_{m}$$
    是同构映射.由此, 上述两种定义直和是同构的.
\end{defn}

如果上述的模都是自由模, 那么它们的基共同组成 $M$的基, 这与线性空间版本并无二致.从而我们可以通过直和分解来研究模的结构, 
\begin{thm}
    主理想整环上的有限生成自由模的子模也是有限生成自由模
\end{thm}
\begin{proof}

    $R$ 是主理想整环,  $M$ 是 $\mathrm{rank}(M) = n$ 的有限生成 $R$ -模,  存在基 $(e_{1},\dots,e_{n})$ 用归纳法: $n =1$ 时,  $M \cong R$, 其上的子模就是 $R$ 的理想, 记 $N = RI$ . 由于 $R$ 是主理想整环, 故 $I =ie_{1}$, 得 $N = R(ie_{1})$, 故 $N$ 是以 $ie_{1}$ 为基的自由模.
假设对秩小于 $n$ 的自由模上述结果都成立, 考虑秩 $n$ 的情况,  
\vspace*{2em}
% https://q.uiver.app/#q=WzAsNSxbMSwwLCJNIl0sWzAsMCwiTiJdLFswLDEsIk4vTlxcY2FwIFJlXzEiXSxbMSwxLCJNL1JlXzEgIl0sWzIsMSwiUmVfMiBcXG9wbHVzXFxjZG90c1xcb3BsdXMgUmVfbiJdLFsxLDBdLFsxLDJdLFswLDNdLFsyLDNdLFszLDQsIlxcY29uZyIsMV1d
\[\begin{tikzcd}
	N & M \\
	{N/N\cap Re_1} & {M/Re_1 } & {Re_2 \oplus\cdots\oplus Re_n}
	\arrow[from=1-1, to=1-2]
	\arrow[from=1-1, to=2-1]
	\arrow[from=1-2, to=2-2]
	\arrow[from=2-1, to=2-2]
	\arrow["\cong"{description}, from=2-2, to=2-3]
\end{tikzcd}\]

若 $N \cap R e_{1} =\{ 0 \}$, 则由归纳假设,  $N = N/ R e_{1}\cap N$ 是有限生成自由模.

若 $N \cap R e_{1} \neq\{ 0 \}$, 由于 $N \cap R e_{1} \subset R e_{1}$, 由归纳假设, $N \cap R e_{1}=Ri e_{1}, \ i \in R$, $N / N \cap R e_{1}$ 也是有限生成自由模, 其基 $\{ [u_{1}],\dots,[u_{k}] \}$, 对于任意的 $w \in N$,  $[w] \in N / N \cap R e_{1}$,
$$[w] =r_{1}[u_{1}]+\dots +r_{k}[u_{k}]$$
即
$$w -r_{1}u_{1}-\dots-r_{k}w_{k}\in R i e_{1}$$
从而 $w$ 被 $\{ u_{1},\dots,u_{k},e_{1} \}$ 线性表出, 这些显然线性无关, 从而是 $N$ 的基, 即 $N$ 是有限生成的自由模.  
\end{proof}


我们再引入最后一个引理,然后来到这次的核心定理.

\begin{lemma}
    主理想整环 $R$上的有限生成自由模 $M$,设 $N$是其子模, 则 $M$存在恰当的基 $(u_1, \ldots ,u_{n})$和 $R$中元素 $a_{i}, 1\leq i\leq m$ 满足
    $$a_{1}|a_{2}|\dots|a_{m}$$ 使得 $(a_1u_1, \ldots ,a_{m}u_{m})$为 $N$ 的基.
\end{lemma}
\begin{proof}
    设 $M$的基 $(e_1, \ldots ,e_{n})$, 以及生成$N$的向量组 $(v_1, \ldots ,v_{k})$,有过渡矩阵 $C\in  M_{n\times k}(R) $:
    \[
    (v_1, \ldots ,v_{k}) = (e_1, \ldots ,e_{n})C
    \]
    由线性代数的知识, 对于矩阵 $C\in M_{n\times k}(R)$,存在可逆矩阵 $P \in GL_{n}(R), Q \in GL_{k}(R)$ ,使得
    \[
    PCQ = \begin{pmatrix} a_1  &  &  &0  \\  & \ddots &  &  \\  &  & a_{m} &  \\ 0 &  &  & 0 \\\end{pmatrix} =:D
    \]
    其中 $a_{i}$满足定理要求. 从而$$(v_1, \ldots ,v_{k})Q =(e_1, \ldots ,e_{n})P^{-1} D$$
    可知 在新基 $(e_1, \ldots ,e_{n})P^{-1} =(u_1, \ldots ,u_{n})$下有$N$的基(容易验证是线性无关的) $$(v_1, \ldots ,v_{k})Q = (a_1u_1, \ldots ,a_{m}u_{m})$$ 从而得证.

    由证明过程知上述 $a_{i}$在相差一个可逆元的情况下是唯一的. 为后续简便起见, 特别的,我们考虑 $R = \mathbb{K}[x]$,那么这些 $a_{i}$就是首项相差一个系数下唯一的多项式.
\end{proof}
\begin{thm}
    不变因子分解:  存在 $R^{n}$ 中的基 $u_{1},\dots,u_{n}$, 和 $a_{i}, 1\leq i\leq m$ 满足
 $$a_{1}|a_{2}|\dots|a_{m}$$ 使得$(a_1u_1, \ldots ,a_{m}u_{m})$为 $\mathrm{ker}\pi _{w}$ 的基, 从而分解 $M = R^{n} / \,\mathrm{ker}\pi _{w}$ 得到

$$M = \frac{Ru_{1}\oplus Ru_{2}\oplus \dots \oplus Ru_{m}\oplus R^{r}}{\braket{ a_{1}} u_{1}\oplus \braket{ a_{2}} u_{2}\oplus \dots \oplus \braket{ a_{m}} u_{m}}\cong \frac{R}{\braket{ a_{1}}} \oplus \dots \oplus \frac{R}{\braket{ a_{m}} }\oplus R^{r}$$


证明:考虑映射 $\sigma$
$$w = w_{1}+\dots+w_{n} \mapsto ([w_{1}], \dots,[w_{m}], w_{m+1},\dots,w_{n})$$
容易验证 $\mathrm{ker}\sigma= \braket{ a_{1}} u_{1}\oplus \braket{ a_{2}} u_{2}\oplus \dots \oplus \braket{ a_{m}} u_{m}$ .


\end{thm}


\begin{thm}
    环 $R$中有一组互素的理想 $a_1, \ldots ,a_{m} \subseteq R$,则有
    \[
    a_1\cdot a_2 \cdots a_{m} = a_1 \cap a_2\cap \cdots\cap a_{m}
    \]
    并有同构
    \[
    R / a_1\cdot a_2\cdots a_{m} \cong \frac{R}{a_1} \oplus \cdots \oplus  \frac{R}{a_{m}}
    \]
\end{thm}
\begin{proof}
    映射 
    \[
    R \to \frac{R}{a_1} \oplus \cdots \oplus  \frac{R}{a_{m}} , \ x \mapsto ([x]_{a_1},[x]_{a_2}, \ldots ,[x]_{a_{m}})
    \]
    显然其核为 $ a_1\cdot a_2\cdots a_{m}$,从而由定理1得证.
\end{proof}

应用中国剩余定理, 通过对 $a_{i} = up_1^{l_{1}}\cdots p^{l_{k}}_{k}$的分解(其中 $p _{i}$是互素的理想),从而我们可以对 $\frac{R}{\braket{a_i}}$进一步分解.即
\[
R / p_1^{l_{1}}\cdots p^{l_{k}}_{k} \cong \frac{R}{p_1^{l_1}}\oplus \cdots \oplus \frac{R}{p_{k}^{l_{k}}}
\]
\vspace*{3em}


接下来我们将上述结构推广到 $n$ 维线性空间中, 从而得到结论: 每个复矩阵 $A$ 都相似于它的 Jordan 标准型.

对数域 $\mathbb{K}$ 上的有限维线性空间 $V$ 上的一个线性映射 $\psi$, 我们定义数乘映射
$$\mathbb{K}[x]\times V \to  V, \ \ (x,v) \mapsto \psi(v)$$
从而对 $a \in \mathbb{K}[x]$, 
$$a \cdot v = (a_{0} +a_{1}\psi+\dots+a_{n}\psi^{n})(v)$$
这样的数乘是可交换的, 通过如此定义, 给定一个线性空间 $V$ 上的线性映射, 我们就可以将其视作一个 $\mathbb{K}[x]$ -模 $V_{\psi}$,  其上的子模就是 $\psi$不变子空间.我们知道这样的模的结构:  $V$ 的一个基 $(e_{1},\dots,e_{n})$ 在 $\mathbb{K}[x]$ 上生成 $V_{\psi}$, 从而由初等因子分解, 
$$V \cong \frac{\mathbb{K}[x]}{p_{1}^{m_{1}}} \oplus \dots \oplus \frac{\mathbb{K}[x]}{p_{k}^{m_{k}}}\oplus \mathbb{K}[x]^{r}$$
由于 $\mathbb{K}[x]^{r}$ 视为 $\mathbb{K}$ 上的线性空间是无限维的, 从而 $r = 0$. 我们考虑 $\mathbb{K} = \mathbb{C}$的特殊情况,其上的素因子只能是一次多项式,从而得到
\[
V \cong \frac{\mathbb{K}[x]}{(\lambda-\lambda_1)^{m_{11}}} \oplus \frac{\mathbb{K}[x]}{(\lambda-\lambda_1)^{m_{12}}} \oplus \dots\frac{\mathbb{K}[x]}{(\lambda-\lambda_1)^{m_{1t_{1}}}}  \oplus \cdots\frac{\mathbb{K}[x]}{(\lambda-\lambda_{s})^{m_{st_{s}}}}
\]

我们先说明这样的分解在模同构的意义下是唯一的.我们考虑特定的 $\lambda_{s}$,将指数按大小排列,由
\[
(\lambda - \lambda_{s})^{j}M_1 \cong (\lambda - \lambda_{s})^{j}M_2, j\in \mathbb{N}
\]
从而如果某一项不等,则 $j$在下降的过程中存在维数不等情况,使得与同构矛盾.


 现在设$\varphi,\psi$是两个 $V$上的线性变换, 对应矩阵 $A,B$, 其分别诱导线性空间的模如果同构,即同构映射 $U$需要适合数乘,
\[
U(x_{a}v) = x_{b}(Uv) 
\]
得到 $ U(Av) = B(Uv)$
得到 $B = UAU^{-1}$,从而 $A,B$相似.反之亦然. 由此, 我们得到此前的分解给出线性映射相似类的刻画.

由线性代数的知识,以上的 $\lambda_{i}$就是 $\psi$的所有特征值,最后我们得到Jordan标准型.考虑子模($\psi$不变子空间),
\[
    \frac{\mathbb{K}[\lambda]}{(\lambda-\lambda_t)^{m_{t}}}
\]
自然有模中的基 $(\lambda-\lambda_t)^{m_{t-1}}, \ldots ,\lambda-\lambda_{t}, 1$,对应着线性空间中的某些向量构成的基 $e_{m_{t}-1}, \ldots ,e_{1}$.由
\[
\lambda((\lambda-\lambda_t)^{m_{t-1}}, \ldots ,\lambda-\lambda_{t}, 1) = ((\lambda-\lambda_t)^{m_{t-1}}, \ldots ,\lambda-\lambda_{t}, 1)\begin{pmatrix} \lambda_{t} & 1 &  &  \\  & \lambda_{t} & \ddots &  \\  &  &  \ddots&  1\\  &  &  & \lambda_{t} \\\end{pmatrix}
\]
这里注意到 $\lambda (\lambda-\lambda_t)^{m_{t}-1} = (\lambda-\lambda_{t})^{m_{t}} +\lambda_{t} (\lambda- \lambda_t)^{m_{t-1}} $, 前一项在该子模中被模掉了.即 $\lambda (\lambda-\lambda_t)^{m_{t}-1} = \lambda_{t} (\lambda- \lambda_t)^{m_{t-1}}$


由此可知
\[
\psi(e_{m_{t}-1}, \ldots ,e_{1})=(e_{m_{t}-1}, \ldots ,e_{1})\begin{pmatrix} \lambda_{t} & 1 &  &  \\  & \lambda_{t} & \ddots &  \\  &  &  \ddots&  1\\  &  &  & \lambda_{t} \\\end{pmatrix}
\]
按此将所有子模的基合为 $V_{\psi}$的基后得到Jordan标准型
\[
\begin{pmatrix} J_{m_1}(\lambda_1) &  &  &  \\  &  J_{m_2}(\lambda_2)&  &  \\  &  & \ddots &  \\  &  &  &  J_{m_k}(\lambda_k)\\\end{pmatrix}
\]
此标准型在相似关系下的不变.
\newpage



Singular的linalg.lib中有计算Jordan标准型的函数,

\begin{lstlisting}
> LIB"linalg.lib";
> ring R = (complex,2),x,dp;
> matrix A[3][3] = 3,2,1,0,2,1,3,2,4;
> print(jordannf(gauss_nf(A)));
2,0,0,
0,3,0,
0,1,3
\end{lstlisting}

似乎只能接受三角阵, 使用gauss\_nf.


\end{document}